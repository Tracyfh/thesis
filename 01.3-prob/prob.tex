\chapter{Probabilistic methods of surface water detection}
\label{ch4}

\begin{abstract}
	A method has been developed to estimate surface water mask for noisy TOA reflectance images, where water bodies can only be partially visible under thin clouds, haze, fog or snow/ice cover. The presence of these effects increases the need for high accuracy and more frequent observations of surface water bodies, making it essential for rapidly varying surface water bodies such as small reservoirs, wetlands, and floodplains. Monitoring these waterbodies at high frequency is essential to understanding changes in the availability of water for food, conditions of fisheries, ecosystems, and the occurrence of floods.

	Here, a Bayesian Network is used as a framework to introduce the method and, finally, to infer the actual surface water area from partially-visible surface water masks. The method makes use of a high-resolution probability density function, which is estimated using a set of cloud-free satellite images from measured by multiple freely available satellite missions from NASA and ESA. Variability of TOA reflectance values for different land use and atmospheric conditions to estimate cloud and snow detection parameters in an automated manner is explored. The method is validated using in-situ water level measurements.
	
	%\url{http://bit.ly/agu16\_reservoirs}
	
	\begin{center}
		\begin{tikzpicture}[every node/.style={inner sep=0,outer sep=0}]
		\node[draw=none,shade,blur shadow={shadow blur steps=5}] {
			\includegraphics[width=3in]{01.3-prob/figures/prob-title}
		};    
		\end{tikzpicture}
	\end{center}
	
	\textbf{Keywords:} $M_2$, probabilistic graphical models, filling, Bayesian Networks, KDE, inference.
	
\end{abstract}

%% Start the actual chapter on a new page.
\newpage

\section{Introduction}
\dropcap{I}{n} the previous chapters, we saw various methods to detect surface water at high accuracy. However, when a waterbody is partially invisible, for example, due to clouds or missing pixels, these methods are insufficient to estimate the actual surface water. In this case, more advanced methods are required to fill missing pixels. In this chapter, this problem is addressed by making use of probabilistic graphical models, in particular, by formulating the task of inferring the most probable water mask in the form of a simple Bayesian Network. 

The general idea is to combine methods discussed in the previous chapter with the probabilistic estimate of surface water mask using a priori knowledge about surface water. This will be done by generating a joint probability density function using surface water estimates based on images where an accurate surface water mask can be generated. 

This turns our task of detecting surface water from satellite data into a generative class of models, where the most probable surface water mask is generated using density functions, in contrast to the discriminative way of processing of satellite data. 

Additionally, a \gls{PGM} notation is used to discuss the steps required to estimate (or infer) the actual surface water mask from available raw satellite data.

\section{Probabilistic graphical models}

\glspl{PGM} (\cite{koller2009probabilistic}) is a rich framework for encoding probability distributions over complex domains, in the form of univariate or multivariate (joint) distributions. Random variables and their conditional dependencies are then expressed in the form of a graph, where child nodes are usually represented by \glspl{CPD}, either defined analytically or in a tabular form, and root nodes are represented by marginal distributions. Two commonly used types of \glspl{PGM} are \glspl{BN} and \glspl{MRF}. However, many other types of \glspl{PGM} exists today, such as \glspl{CRF}, factor graphs, restricted Boltzmann machines and many others.

\begin{wrapfigure}{r}{0.4\textwidth}
	\begin{center}
		\includegraphics[width=0.3\textwidth]{01.3-prob/figures/bn_example} 
	\end{center}
	\caption{A simple Bayesian network.}
	\label{fig:bn_example}
\end{wrapfigure}

Here, a \gls{BN} notation is used, which is usually expressed in the form of a \gls{DAG}. Consider the example of a simple network as shown in \ref{fig:bn_example}. Here, $X_{1...3}$ nodes represent random variables and the shaded variable $X_2$ indicates a variable that is measured. 

The root nodes of the graph of a BN are defined as marginal distributions, which are in general expressed as single-variable \glspl{PDF}, and all child nodes can then be represented as \glspl{CPD}, which can be expressed as multivariate probability distribution functions. 

This DAG can then be used to conduct statistical inference, for example, to infer the unknown variables $X_3$ and $X_1$ by making assumptions about their distributions, while at the same time, satisfying the available (measured) distribution of $X_2$. The joint distribution of this BN can then be expressed (factorized) as:

\begin{equation}
P{(X_1, X_2, X_3)} = P{(X_1)} P{(X_2 \mid X_1)} P{(X_3 \mid X_1, X_2)}
\end{equation}

or in a general form:

\begin{equation}
P{(X_i, \ldots, X_n)} = \prod_{i=1}^{n} P{(X_i \mid S_i)}
\end{equation}

where $S_i$ represents a set of parents of every node $X_i$.

After defining the model, many types of questions can be answered, by means of direct inference methods, such as conditioning parent variables and computing resulting distributions of dependent variables, marginalization --- deriving marginal distribution by integrating over other variables,  or by applying methods of learning to simulate the graph as a whole and to infer hidden variables.

Many methods exist to perform inference in \glspl{BN}, such as \gls{MCMC}, or \gls{EM}. The main goal of these methods is to estimate unknown distributions from known variables and hypothetical relations between them. Many examples of \glspl{PGM} as well as methods used to conduct inference in \glspl{BN} are given in \cite{koller2009probabilistic}. 

\section{Plate notation}

One of the useful notations used for \glspl{BN} is a \textit{plate notation}, which allows grouping of multiple random variables of a similar type. In this way, more complex models can be expressed by the network, as shown in Figure \ref{fig:plate-example}.

\begin{figure}[H]
	\centering
	\includegraphics[width=1\textwidth]{01.3-prob/figures/plate_example}
	\caption{An example of the use of plate notation (right) for Naive Bayes classifier graphical model (left).}
	\label{fig:plate-example}
\end{figure}

The above graph, in fact, represents a model of a Naive Bayes classifier, where the random variable $w$ represents a distribution of some feature belonging to some class, given a set of independent variables $X_i$ with known conditional distributions $P{(X_i|w)}$. During training, the model variables $X_i$ (conditional distributions) are defined. During inference, the most probable class is selected, for example, by estimating the maximum a posteriori probability.

\section{Discriminative vs. Generative methods}

In machine learning, a generative model is defined as a model where some variable values are generated from a joint probability density function. Generative models differ from discriminative models in that they require the use of joint probability density function to be constructed, while discriminative models provide a way to model only the target variables conditioned on observed variables \citet{ng2002discriminative}. Some examples of generative models include Naive Bayes classifier, Gaussian Mixture Model, and the Hidden Markov Model while examples of discriminative models include Logistic Regression, Support Vector Machines, and Conditional Random Fields. 

The method presented in this chapter can be classified as a generative one, as the joint probability density function is estimated for pixels belonging to water $P(x, y)$. However, it will be estimated in a discriminative way, by combining surface water masks detected from cloud-free images, as was discussed in the previous chapter.

\section{Probabilistic nature of the satellite observations}

Here, the multispectral satellite observations are expressed in the form of a Bayesian network, to allow better reasoning about the methods used to detect surface water changes. Eventhough, standard algorithms used to reconstruct marginal distributions were not applied, the framework is a very rich way to express and explain probabilistic surface water inference methods introduced later in this chapter.

In multispectral remote sensing using satellites, we can think of observed radiance or reflectance values as random variables. In fact, the observed values represent a joint distribution in a multidimensional space (time, space, wavelength). Even though most of the sensors provide 2D surface radiance values, the actual values are reflected from the 3D surface of the Earth. The actual values measured at-the-sensor represent a signal, backscattered from the aerosol particles, and from the ground (frequently represented by non-Lambertian surfaces). The actual path of the light is extremely complex and may be impossible to model in a deterministic way. Interpreting this signal using probabilistic models makes more sense.

For passive multispectral satellite sensors, the most important factors disturbing the resulting radiance values are clouds and topographic effects. While many algorithms exist to detect clouds, modeling complex interaction of cloud and hill shadows, thin clouds, fog or haze is extremely difficult. Additionally, physical parameters of the sensors, such as spatial and radiometric resolution, result in observations of mixed signals, reflected from multiple surfaces, such as different types of soil, water, and vegetation. Furthermore, surface water pixels may be fully or partially covered by snow and ice.

In Figure \ref{fig:pgm} the most important variables influencing the observed signal are shown. Note, that the observed variables are shown as shaded circles and that wavelength, time and space ($x \in \mathbb{R}^2$) are defined using plate notation.

\begin{figure}
	\centering
	\includegraphics[width=1.0\textwidth]{01.3-prob/figures/probabilistic_model} 
	\caption{\gls{BN} of a passive multispectral satellite sensor. Filled circles represent observed variables, with a known distributions. Empty circles indicate latent variables that need to be reconstructed. The variable $L$ defines observed radiances, where values in the distribution are based on the solar parameters, cloud cover, topography, as well as on the actual land use classes occurring at a given location. Plate notation is used to indicate dependency on time, space and wavelength.}
	\label{fig:pgm}
\end{figure}

A more general task of remote sensing can be defined as a classification of the hidden, or latent, variable $Y_{class}$ for every land-use type observed by the satellite. Clouds are introduced as a separate variable $C$, to stress, that it represents a noise and is not mixed with the actual land use type hidden under the clouds or distorted by the cloud shadows. 

To indicate a complex light interaction between clouds and Earth's surface topography, a separate random variable $Z$ is added. It is a random variable, because it is never measured exactly, and may be very noisy for most existing global digital elevation datasets, such as 30m \gls{SRTM} or 3m \gls{NED}. These errors can be very large (sometimes ~50m) and may result in an incorrect interpretation of the observed pixels when used, for example, to model illumination effects, as shown in Chapter \ref{ch2}.

The Sun parameters, $S_\phi$ and $S_\theta$ represent the Sun's azimuth and elevation at the time, when the image was measured, and are usually very accurate, perhaps making it more appropriate to introduce them as parameters instead of random variables.

The Sun's light interacts with clouds and topography, which results in a distortion of the signal, which is expressed by the variables $C_s$, $Z_s$, which may indicate the amount of light scattered and absorbed due to clouds and shadows.

The arrows between these variables represent conditional dependencies. For example, the conditional probability representing the amount of light influenced by cloud shadows can be defined as: $P(C_s|Z,S_\phi,S_\theta,C)$. Here, the cloud shadow variable depends on the topography because the actual illumination usually changes depending on the cloud height, which in turn depends on the actual elevation at that location.

To reconstruct (infer) unknown variables, one can try to simulate the joint probability for this PGM factored as:

\begin{equation}
\begin{split}
\begin{aligned}
P{(L_{\lambda_1}, ... L_{\lambda_N}, C, C_s, Z, Z_s, S_\phi, S_\theta, Y_{class_1} ... Y_{class_M})} &= P(Y_{class_1}) ... P(Y_{class_M}) \\
& P{(C)} P{(S)} P({C_s)} P{(S_\theta)} P{(S_\phi)}  P{(Z_s)} P{(Z)} \\ 
& P{(C_s \mid C,S_\theta,S_\phi,Z)}  P{(Z_s|Z,S_\theta, S_\phi)}
\end{aligned}
\end{split}
\label{eq:bn-joint}
\end{equation}

with $N$ bands $\lambda$ and $M$ classes $class$ defined as:

\begin{equation}
\begin{split}
\begin{aligned}
\lambda &= \{blue, green, red, nir, swir1, swir2, temp\} \\
class &= \{water, snow, sand, vegetation, urban, ...\}
\end{aligned}
\end{split}
\end{equation}

Note that in addition to the class and wavelength dimensions, most of the variables also depend on space and time, which is indicated as indices of the corresponding plates. 

While variables $Y_{class}$ can be simulated as discrete variables, with a corresponding probability mass function representing the probability of a pixel belonging to a given class, we can also think of it as a continuous variable, defined for every class separately and indicating probability of a pixel fraction (varying from 0 to 1) belonging to a given land-use class. Many of the variables in the above equation are known or partially known as well as their respective errors, however, it may be difficult to simulate the whole joint distribution using standard methods (MCMC), mainly because of the amount of data to be processed.

For the water detection task, we are interested in the reconstruction of $P(C_{water})$ distribution, or more generally - as a multivariate distribution $P(C_{water}, x, y)$, marginalized in time and representing the probability of water in every pixel of a given area for some specific period. It is important to note that this can be done only under the assumption that morphology of the waterbody does not change during the selected period. For more complex configurations, this density function should be assumed as a time-dependent function, for example, to represent seasonal surface water variability. 

One of the methods to infer this water distribution, as well as other latent variables in the above graphical model, is to estimate them in a discriminative way, directly from the observed reflectance values measured by satellite sensors. With sufficient measurements and under the assumption that morphology of the waterbody does not change significantly, we should be able to condition our joint distribution \ref{eq:bn-joint} into subsets where clouds are present and absent. Also, instead of integrating this joint distribution, marginalizing it to estimate $P(C_{water}, x, y)$, it can be directly computed by making use of the methods described in Chapter \ref{ch3} and estimating the frequency of pixels to be covered by water or land.

An alternative could be to fully simulate the above graphical model by introducing some analytical models, like FMask or topographic correction formulas from Chapter \ref{ch2} to simulate the \glspl{CPD} representing cloud and hill shadows. After the reconstruction, standard methods used for \glspl{BN} can be applied to infer all other unknown land-use class variables $P(C_{class})$.

%\url{http://sci-hub.cc/10.1111/j.1467-9876.2010.00739.x}
%\url{https://courses.engr.illinois.edu/cs543/sp2011/lectures/Lecture\%2012\%20-\%20MRFs\%20and\%20Graph\%20Cut\%20Segmentation\%20-\%20Vision_Spring2011.pdf}

\subsection{Sampling spectral reflectance values from multi-temporal Landsat images for water and land pixels}

In many cases, distributions of the marginal variables introduced in Figure \ref{fig:pgm} are unknown and may represent very dynamic processes occurring at the same time. To demonstrate this complexity, N=100 locations are sampled for the area outlined in Figure \ref{fig:study-area-PC}, and then, using only an area covered by surface water, in the middle of the reservoir. Because surface water of the reservoir changes in time, but also because image registration may be imperfect, it is hard to ensure that all sampled values correspond to water and clouds only. However, it should be sufficient to demonstrate the variability of the processes over a period of 30 years.

\begin{figure}[H]
	\includegraphics[width=0.8\textwidth,center]{01.3-prob/figures/map-ProsserCreekReservoir-sampling-area}
	\centering
	\captionsetup{justification=centering}
	\caption{Sampling area used for scatter plots, Prosser Creek Reservoir, CA, USA}
	\label{fig:study-area-PC}
\end{figure}

\begin{figure}[H]
	\includegraphics[width=1.0\textwidth,left]{01.3-prob/figures/sampled_water_clouds_snow}
	\includegraphics[width=1.0\textwidth,center]{01.3-prob/figures/sampled_scalebar}

	\caption{TOA reflectance values sampled over the study area but covering mainly water pixels. Scatter plot points are colored as a false-color combination: [swir1, nir, green]. The scalebar indicates probability after applying the \gls{KDE} method.}
	\label{fig:prob-sampled-water-cloud-snow}
\end{figure}

\begin{figure}[H]
	\includegraphics[width=1.0\textwidth,left]{01.3-prob/figures/sampled_water_soil_clouds_snow}
	\includegraphics[width=1.0\textwidth,center]{01.3-prob/figures/sampled_scalebar}
	
	\caption{TOA reflectance values sampled over study area (water and soil pixels) from all Landsat images available for that area (N=930).}

	\label{fig:prob-sampled-water-soil-cloud-snow}
\end{figure}

\begin{figure}[H]
	\includegraphics[width=1.0\textwidth,left]{01.3-prob/figures/sampled_water_soil}
	\includegraphics[width=1.0\textwidth,center]{01.3-prob/figures/sampled_scalebar}

	\caption{TOA reflectance values after conditioning on cloud and snow cover.}
	
	\label{fig:prob-sampled-water-soil}
\end{figure}


The resulting scatter matrix plots, colored as a false-color image using swir1, nir, and green bands, are shown in figures \ref{fig:prob-sampled-water-cloud-snow} and \ref{fig:prob-sampled-water-soil-cloud-snow}, for water-only pixels and water plus soil pixels. Additionally, the density of the pixels for every scatter plot in the lower triangle is analyzed. The reason for this exercise is to try to see if different classes can be determined from these distributions or at least to identify the most frequently occurring values in the case of water, soil, clouds or snow. It is interesting to see that for this area, the distance between the most frequently occurring values, which can be used to separate water and land, correspond to nir and green bands. However, also the temperature band shows a lot of variability, where temperature values are formed by the presence of snow, soil, and water in pixels (three peaks in the histogram of the temperature band) as shown in the directed graph above. The Figure \ref{fig:prob-sampled-water-cloud-snow} represents a joint distribution, conditioned on some of the variables, in this case, on $P(C_{water}=1)$. As can also be seen, reflectance values for clouds and snow vary significantly in contrast to lower reflectance values in the case of water or soil. 

By sampling values in the center of the reservoir and the area covering water and the nearby land, some of the parameters can be automatically reconstructed from these distributions to be used in combination with classical remote sensing methods for cloud detection, such as ACCA or FMask. Adjusting the actual thresholds used in these methods with more objective parameters, it should be possible to automatically identify fully cloud-free images, and then use them to simulate the marginal distribution representing water occurrence. 

Here, this approach is used to adjust threshold values used to estimate clouds and snow scores, to identify the most cloud-free images over the reservoir area (in fact, conditioning the joint distribution for $L$ on a $C=0$). These cloud-free images are used to reconstruct the marginal distribution $P(C_{water}, x, y))$ for a given time interval, assuming that the water dynamics under the clouds looks similar to when there are clouds.

\section{Probabilistic filling of missing pixels in surface water detection}

One of the ways to solve the problem of filling missing pixels, or to remove the less probable false-positive water pixels is to make use of the a priori information about surface water dynamics. By adjusting cloud (or snow/ice) detection algorithms with local statistics, we can identify images where the number of bad or missing pixels is small, resulting in very accurate water masks. The figures \ref{fig:prob-water-isolines} and \ref{fig:prob-water-density} show edges for several of these water masks and a 2D probability density image estimated from these water masks.

\begin{figure}[H]
	\centering
	\includegraphics[width=1\textwidth]{01.3-prob/figures/contour_lines}
	\caption{Surface water mask detected from cloud-free images (multiple sensors). The images were resampled using bicubic interpolation before the actual mask was detected. This has resulted in a better final water occurrence.}
	\label{fig:prob-water-isolines}
\end{figure}


\begin{figure}[H]
	\centering
	\includegraphics[width=0.6\textwidth]{01.3-prob/figures/water_occurrence}
	\caption{Water occurrence (prior) estimated from cloud-free images measured by multiple satellites (ASTER, Sentinel-2, and Landsat)}
	\label{fig:prob-water-density}
\end{figure}

\begin{figure}[H]
	\centering
	\includegraphics[width=1\textwidth]{01.3-prob/figures/method-diagram}
	\caption{A high-level workflow of the surface water detection method}
	\label{fig:prob-workflow}
\end{figure}

The method to fill surface area for missing pixels is based on the use of this probability density image to infer water mask, employing Bayes' rule. The actual algorithm involves a few more steps, where the reconstructed water \gls{PDF} is also used during edge detection (less probable water edges are removed). The resulting algorithm, outlined in Figure \ref{fig:prob-workflow} allows for the automated estimation of reservoir surface area using both cloud-free and partially cloud-free images. 

In fact, the inference step of the algorithm can be schematically visualized as in \ref{fig:prob-diagram}, where on the left side, the original graph is shown, with all irrelevant variables removed. 

\begin{figure}[H]
	\centering
	\includegraphics[width=0.4\textwidth]{01.3-prob/figures/probabilistic_model_single_scene}
	\caption{An example of a BN for a single scene, assuming marginal distribution on the left estimated during the previous step.}
	\label{fig:prob-diagram}
\end{figure}

During the first step, where the water probability density function was estimated, the Landsat 7 and Sentinel-1 images were excluded, because of SLC-OFF errors and because the \gls{SAR} water detection algorithm may be less reliable, and because of effects like Bragg scattering, resulting in false-negative pixels.

In many cases, slightly better results are achieved for noise-free pixels when compared to probabilistic estimate (our initial prior estimate is never perfect). An additional step was added to choose between a generative surface water estimate and the one performed using discriminative methods. The choice between different water masks detected using generative (density function) and discriminative (dynamic thresholding) methods is made based on the quality of the pixels, which depends on cloud and snow scores. For very hilly topographies, it may be necessary to make use of $Z_s$ as well, to avoid misclassification of hill shadows as water pixels.

\section{Application to the reservoir surface water area reconstruction}

\begin{figure}
	\centering
	\includegraphics[width=1.0\textwidth]{01.2-water-detection/figures/PC_NDWI_3_Otsu_prior}
	\caption{Canny/Otsu, probabilistic update}
	\label{fig:r1_canny_otsu_nb}
\end{figure}

\begin{figure}
	\centering
	\includegraphics[width=1.0\textwidth]{01.2-water-detection/figures/PC_NDWI_2_Otsu_HAND_prior}
	\caption{Canny/Otsu, HAND < 15m, probabilistic update}
	\label{fig:r1_canny_otsu_hand_nb}
\end{figure}

\begin{figure}
	\centering
	\includegraphics[width=1.0\textwidth]{01.2-water-detection/figures/PC_NDWI_4_Otsu_HAND_prior_fill}
	\caption{Canny/Otsu, probabilistic update, HAND < 15m, fill}
	\label{fig:r1_canny_otsu_hand_nb_fill}
\end{figure}

\begin{figure}
	\centering
	\includegraphics[width=1.0\textwidth]{01.2-water-detection/figures/PC_NDWI_4_Otsu_HAND_topo_prior_fill}
	\caption{topographic, Canny/Otsu, probabilistic update, HAND < 15m, fill}
	\label{fig:r1_canny_otsu_topo_hand_nb_fill}
\end{figure}

A hybrid, stepwise method, combining image processing algorithms and remote sensing methods would be more appropriate for the accurate estimation of the water mask.

\newpage


\section{Measuring performance of surface water detection with filling applied}

To evaluate the overall multi-step algorithm performance, where a combination of the dynamic thresholding and the generative estimation of surface water mask were involved, the final surface water area values were compared with the surface water area estimates computed using the statistical model introduced in Chapter \ref{ch3}. 

Additionally, surface water masks not passing the quality criteria were excluded. The quality score was defined as a combination of multiple parameters, such as the fraction of false-positive, false-negative pixels, and the number of cloud and snow pixels. 

The final surface water area time series, together with the original surface area, estimated from only optical satellite images and NDWI=0, are shown in Figure \ref{fig:prob-results}.

To further explore the quality of the estimates, a residuals analysis is performed for the final results, comparing them to the estimates based on the model constructed for the reservoir. We can see from the figures, that the errors are mostly normally distributed, $RMSE=0.059km^2$, indicating good performance of the algorithm.

In the case of cloud presence, the actual threshold variability of the spectral indices makes them less applicable, resulting in much larger variation of the surface water mask estimates. Use of the step-wise approach introduced in the last two chapters provides a way to automatically detect surface water with a very high degree of accuracy.

\begin{figure}[H]
	\centering
	\includegraphics[width=1.0\textwidth]{01.3-prob/figures/results}
	\caption{Comparison of estimated surface water area $A$ vs. simulated surface water $A*$, based on the linear regression constructed previously. Scatter plots (a, b) show results of the surface water detection algorithm, without (a) and with (c) probabilistic filling. Time series charts (b, d) correspond to estimated (dots) and measured surface water area values.}
	\label{fig:prob-results}
\end{figure}

\section{Analysis of residuals and outliers}

\begin{figure}[H]
	\centering
	\includegraphics[width=1.0\textwidth]{01.3-prob/figures/residuals}
	\caption{Analysis of residuals (surface area estimated from satellite images vs. surface area estimated from the linear model). The top chart (a) compares time series estimated from water level measurements and satellite photos (dots), as well as outliers ($\epsilon > \sigma$). Chart (b) shows residuals values, assuming surface water area A* as the ground truth. Scatter plots (c, d) show a comparison of measured water level and estimated surface water area (c) vs. water levels and A/A* plot (d). The chart (e) indicates histogram of residual values (<3\%).}
	\label{fig:prob-results}
\end{figure}

%\section{Comparison with the JRC surface water mask dataset}
%\textbf{compare monthly surface water changes estimated from JRC dataset %(scatter plots, time series and residual analysis)}

\section{Difficult surface water detection examples}

To demonstrate the method performance, figures \ref{fig:prob-guess1}, \ref{fig:prob-guess2}, \ref{fig:prob-guess3} demonstrate a few examples where the surface water mask estimate is particularly difficult. Using the method presented here, an almost perfect water mask can be reconstructed, when compared to the observed water level values. 

The first example in Figure \ref{fig:prob-guess1} illustrates the algorithm performance for the case in which the reservoir is almost fully covered by clouds, and only a small area in the upper part of the reservoir can be seen. Because these pixels are located in a very flat area, where water surface area changes correlate with changes in water levels, accurate estimates of the actual surface area could be made by utilizing the density function constructed above. This case may result in larger errors in some reservoirs, where large morphological changes may take place in the river. In this case, our prior distribution, used to provide posterior estimates, is less informative. In this case, it would be more appropriate to utilize the temporal variability of the surface area. This is relatively easy to achieve by constructing the water occurrence distribution as three-dimensional, instead of two-dimensional - $P(C_{water}) = f(x,y,t)$, where the probability of water at a given location may change in time as well. 

\afterpage{%
	\begin{figure}
		\centering
		\includegraphics[width=0.8\textwidth]{01.3-prob/figures/guess1_clipped.png}
		\caption{Surface water reconstructed from a small amount of visible water at the edge of the reservoir. Visible water pixels are located in a dynamic part of the reservoir.}
		\label{fig:prob-guess1}
	\end{figure}
	
	\begin{figure}
		\centering
		\includegraphics[width=0.8\textwidth]{01.3-prob/figures/guess2_clipped.png}
		\caption{Surface water reconstructed from a Sentinel-2 image with a very dark cloud shadow present. False-positive surface water detected during first pass (\textcolor{red}{red}) is corrected after the use of Bayesian update step.}
		\label{fig:prob-guess2}
	\end{figure}
	
	\begin{figure}
		\centering
		\includegraphics[width=0.8\textwidth]{01.3-prob/figures/guess3_clipped.png}
		\caption{Surface water mask reconstructed from Sentinel-1 \gls{SAR} image. A very limited number of pixels are visible in a steep part of the reservoir.}
		\label{fig:prob-guess3}
	\end{figure}
	\clearpage
}

Figure \ref{fig:prob-guess2} shows a case where the original surface water mask estimate is incorrect with regards to the very dark cloud shadow pixels. The image was acquired at a relatively low sun elevation, resulting in lower reflectance values, with a very evident cloud shadow. In fact, looking at the values \gls{NDWI} of the pixels, corresponding to water and cloud shadow only, it was impossible to detect the surface water mask reliably. In this situation, as well as in the case of SLC-OFF images, the surface water mask can be accurately estimated. This claim holds true because a strong correspondence between water occurrence and detected surface water edges is expected to occur.

A slightly more difficult case can be seen in Figure \ref{fig:prob-guess3}, where a small part of the Sentinel-1 \gls{SAR} backscatter intensity image is used to estimate surface water. The challenge here lies in the fact that this part of the reservoir is relatively steep, resulting in a much higher chance of an incorrect surface water area estimate. Another challenge here is related to the presence of a very dark edge present in the Sentinel-1 image, which was excluded by eliminating low entropy areas along the image edges. 

Density-based methods may be extremely useful when the waterbody is partially covered by clouds, cloud shadows, or when only part of a waterbody is visible, due to limited swath or due to sensor artifacts (SLC-OFF for Landsat 7). 

\section{Conclusions and Discussion}

In Chapter \ref{ch3}, it was demonstrated that automated surface water detection from multiple multispectral missions is feasible, however, the surface water detection from noisy satellite images remained difficult. Where such noisy images are frequently occurring (e.g. in predominantly clouded regions) or where high frequency observations are required, the applicability of the method presented in Chapter \ref{ch3} is limited and more advanced methods are required. 

In this chapter, it has been shown that even when a small part of a surface water body is observed and the remainder obscured, the rest of the surface water body can be inferred by analyzing historical imagery in which accurate estimates of the surface water body were established. The accuracy of the final surface water mask depends on how informative the observed surface water area is, which directly depends on how many changes (inter- or intra-annual) occur to the surface water at a given location.

The historical imagery is introduced in the water classification algorithm through Bayesian inference. However, this also assumes independence of the observations from each other, resulting in overestimation or underestimation of surface water area in several cases, such as very noisy images with multiple types of noise present: snow, clouds, shadows or very noisy backscatter data. To overcome these problems, many other methods can also be applied such as Hidden Markov Models (HMM), where transition probabilities can be estimated based on multiple historical observations.

Essentially, the method presented in this chapter allows enhancing the spatial resolution of the final estimated water mask. This is done by combining surface water masks from multiple sensors, with the spatial resolution varying between 10m (Sentinel-2) up to 30m (for Landsat). The final density function was estimated at a resolution of 5m combined with the bicubic resampling applied to the images to interpolate values. Eventhough the actual spectral signature gets damaged at the sub-pixel level, it was possible to generate very accurate surface water mask estimates due to the use of local dynamic thresholding methods introduced in Chapter \ref{ch3}.

The framework presented in this chapter can also be used in combination with more general image segmentation methods instead of local thresholding. Alternatives may include image segmentation algorithms such as local k-means (SLIC, \citet{achanta2012slic}), combined with methods such as Markov Random Fields (MRF) or Conditional Random Fields (CRF) to infer the actual land use types based on the spatial neighborhood and temporal variability of reflectance values.
