\chapter*{Summary}
\addcontentsline{toc}{chapter}{Summary}
\setheader{Summary}

%PF>>
\dropcap{T}{his} thesis studies automated methods of surface water detection from satellite imagery. Multiple existing methods are explored, discussed, and some new algorithms are introduced to allow very accurate detection of surface water and surface water changes. 

The methods range in applicability from the local level to global, and from detecting high-frequency changes to low-frequency changes. Their trade-offs regarding the accuracy and applicability of the surface water detection methods are also discussed.

Several applications are presented to test the introduced methods. One of the studies focuses on a long-term global surface water change detection over the past 30 years at 30m resolution. The other application looks at the generation of a permanent surface water mask for Murray-Darling River Basin in Australia. Additionally, an in-depth validation for a small reservoir in California, USA, is presented to demonstrate the performance of the new methods. 

The algorithms discussed in the thesis were applied and tested to process both passive optical multispectral and active synthetic aperture radar (SAR) satellite data. Combining data from all freely available satellite sensors requires harmonizations of the satellite data, but also, a significant compute resources. In this thesis, Google Earth Engine parallel processing platform was used to perform most of the experiments.

We will see, that when studying surface water dynamics, the best results can be achieved by combining discriminative and generative methods of surface water detection. This way, the surface water can also be detected from satellite images where surface water is only partially visible. 

In the thesis, top-of-atmosphere reflectance images are used to detect surface water. The atmospheric correction is not required when dynamic local thresholding methods are used to detect surface water.

%PF<<<